% Options for packages loaded elsewhere
\PassOptionsToPackage{unicode}{hyperref}
\PassOptionsToPackage{hyphens}{url}
%
\documentclass[
]{book}
\usepackage{lmodern}
\usepackage{amsmath}
\usepackage{ifxetex,ifluatex}
\ifnum 0\ifxetex 1\fi\ifluatex 1\fi=0 % if pdftex
  \usepackage[T1]{fontenc}
  \usepackage[utf8]{inputenc}
  \usepackage{textcomp} % provide euro and other symbols
  \usepackage{amssymb}
\else % if luatex or xetex
  \usepackage{unicode-math}
  \defaultfontfeatures{Scale=MatchLowercase}
  \defaultfontfeatures[\rmfamily]{Ligatures=TeX,Scale=1}
\fi
% Use upquote if available, for straight quotes in verbatim environments
\IfFileExists{upquote.sty}{\usepackage{upquote}}{}
\IfFileExists{microtype.sty}{% use microtype if available
  \usepackage[]{microtype}
  \UseMicrotypeSet[protrusion]{basicmath} % disable protrusion for tt fonts
}{}
\makeatletter
\@ifundefined{KOMAClassName}{% if non-KOMA class
  \IfFileExists{parskip.sty}{%
    \usepackage{parskip}
  }{% else
    \setlength{\parindent}{0pt}
    \setlength{\parskip}{6pt plus 2pt minus 1pt}}
}{% if KOMA class
  \KOMAoptions{parskip=half}}
\makeatother
\usepackage{xcolor}
\IfFileExists{xurl.sty}{\usepackage{xurl}}{} % add URL line breaks if available
\IfFileExists{bookmark.sty}{\usepackage{bookmark}}{\usepackage{hyperref}}
\hypersetup{
  pdftitle={ndexr},
  pdfauthor={Freddy R. Drennan},
  hidelinks,
  pdfcreator={LaTeX via pandoc}}
\urlstyle{same} % disable monospaced font for URLs
\usepackage{color}
\usepackage{fancyvrb}
\newcommand{\VerbBar}{|}
\newcommand{\VERB}{\Verb[commandchars=\\\{\}]}
\DefineVerbatimEnvironment{Highlighting}{Verbatim}{commandchars=\\\{\}}
% Add ',fontsize=\small' for more characters per line
\usepackage{framed}
\definecolor{shadecolor}{RGB}{248,248,248}
\newenvironment{Shaded}{\begin{snugshade}}{\end{snugshade}}
\newcommand{\AlertTok}[1]{\textcolor[rgb]{0.94,0.16,0.16}{#1}}
\newcommand{\AnnotationTok}[1]{\textcolor[rgb]{0.56,0.35,0.01}{\textbf{\textit{#1}}}}
\newcommand{\AttributeTok}[1]{\textcolor[rgb]{0.77,0.63,0.00}{#1}}
\newcommand{\BaseNTok}[1]{\textcolor[rgb]{0.00,0.00,0.81}{#1}}
\newcommand{\BuiltInTok}[1]{#1}
\newcommand{\CharTok}[1]{\textcolor[rgb]{0.31,0.60,0.02}{#1}}
\newcommand{\CommentTok}[1]{\textcolor[rgb]{0.56,0.35,0.01}{\textit{#1}}}
\newcommand{\CommentVarTok}[1]{\textcolor[rgb]{0.56,0.35,0.01}{\textbf{\textit{#1}}}}
\newcommand{\ConstantTok}[1]{\textcolor[rgb]{0.00,0.00,0.00}{#1}}
\newcommand{\ControlFlowTok}[1]{\textcolor[rgb]{0.13,0.29,0.53}{\textbf{#1}}}
\newcommand{\DataTypeTok}[1]{\textcolor[rgb]{0.13,0.29,0.53}{#1}}
\newcommand{\DecValTok}[1]{\textcolor[rgb]{0.00,0.00,0.81}{#1}}
\newcommand{\DocumentationTok}[1]{\textcolor[rgb]{0.56,0.35,0.01}{\textbf{\textit{#1}}}}
\newcommand{\ErrorTok}[1]{\textcolor[rgb]{0.64,0.00,0.00}{\textbf{#1}}}
\newcommand{\ExtensionTok}[1]{#1}
\newcommand{\FloatTok}[1]{\textcolor[rgb]{0.00,0.00,0.81}{#1}}
\newcommand{\FunctionTok}[1]{\textcolor[rgb]{0.00,0.00,0.00}{#1}}
\newcommand{\ImportTok}[1]{#1}
\newcommand{\InformationTok}[1]{\textcolor[rgb]{0.56,0.35,0.01}{\textbf{\textit{#1}}}}
\newcommand{\KeywordTok}[1]{\textcolor[rgb]{0.13,0.29,0.53}{\textbf{#1}}}
\newcommand{\NormalTok}[1]{#1}
\newcommand{\OperatorTok}[1]{\textcolor[rgb]{0.81,0.36,0.00}{\textbf{#1}}}
\newcommand{\OtherTok}[1]{\textcolor[rgb]{0.56,0.35,0.01}{#1}}
\newcommand{\PreprocessorTok}[1]{\textcolor[rgb]{0.56,0.35,0.01}{\textit{#1}}}
\newcommand{\RegionMarkerTok}[1]{#1}
\newcommand{\SpecialCharTok}[1]{\textcolor[rgb]{0.00,0.00,0.00}{#1}}
\newcommand{\SpecialStringTok}[1]{\textcolor[rgb]{0.31,0.60,0.02}{#1}}
\newcommand{\StringTok}[1]{\textcolor[rgb]{0.31,0.60,0.02}{#1}}
\newcommand{\VariableTok}[1]{\textcolor[rgb]{0.00,0.00,0.00}{#1}}
\newcommand{\VerbatimStringTok}[1]{\textcolor[rgb]{0.31,0.60,0.02}{#1}}
\newcommand{\WarningTok}[1]{\textcolor[rgb]{0.56,0.35,0.01}{\textbf{\textit{#1}}}}
\usepackage{longtable,booktabs}
\usepackage{calc} % for calculating minipage widths
% Correct order of tables after \paragraph or \subparagraph
\usepackage{etoolbox}
\makeatletter
\patchcmd\longtable{\par}{\if@noskipsec\mbox{}\fi\par}{}{}
\makeatother
% Allow footnotes in longtable head/foot
\IfFileExists{footnotehyper.sty}{\usepackage{footnotehyper}}{\usepackage{footnote}}
\makesavenoteenv{longtable}
\usepackage{graphicx}
\makeatletter
\def\maxwidth{\ifdim\Gin@nat@width>\linewidth\linewidth\else\Gin@nat@width\fi}
\def\maxheight{\ifdim\Gin@nat@height>\textheight\textheight\else\Gin@nat@height\fi}
\makeatother
% Scale images if necessary, so that they will not overflow the page
% margins by default, and it is still possible to overwrite the defaults
% using explicit options in \includegraphics[width, height, ...]{}
\setkeys{Gin}{width=\maxwidth,height=\maxheight,keepaspectratio}
% Set default figure placement to htbp
\makeatletter
\def\fps@figure{htbp}
\makeatother
\setlength{\emergencystretch}{3em} % prevent overfull lines
\providecommand{\tightlist}{%
  \setlength{\itemsep}{0pt}\setlength{\parskip}{0pt}}
\setcounter{secnumdepth}{5}
\usepackage{booktabs}
\ifluatex
  \usepackage{selnolig}  % disable illegal ligatures
\fi
\usepackage[]{natbib}
\bibliographystyle{apalike}

\title{ndexr}
\author{Freddy R. Drennan}
\date{2020-12-26}

\begin{document}
\maketitle

{
\setcounter{tocdepth}{1}
\tableofcontents
}
\hypertarget{about-me}{%
\chapter{About Me}\label{about-me}}

You probably know me from somewhere. My name is Freddy Drennan and I have been in and out of data related projects for the past five years. I initially fell in love with the \texttt{R} programming language and best practices in programming development - unit testing, well documented code, quality over quantity type stuff.

I have wanted to break out of my programming rut for a while. Having devoured R, I want a bigger challenge. I'll be blogging about C, C++, and Rust in the coming weeks - taking notes about what I have learned as well and track my own growth.

Thanks for checking this out!

\begin{Shaded}
\begin{Highlighting}[]
\FunctionTok{print}\NormalTok{(}\StringTok{\textquotesingle{}Hey, yalls, thanks again!\textquotesingle{}}\NormalTok{)}
\end{Highlighting}
\end{Shaded}

\begin{verbatim}
## [1] "Hey, yalls, thanks again!"
\end{verbatim}

\hypertarget{c}{%
\chapter{C}\label{c}}

\hypertarget{basic-calculator}{%
\section{Basic Calculator}\label{basic-calculator}}

\begin{Shaded}
\begin{Highlighting}[]
\PreprocessorTok{\#include }\ImportTok{\textless{}stdio.h\textgreater{}}

\DataTypeTok{int}\NormalTok{ main() \{}
    \CommentTok{/*print Fahrenheit{-}Celsius table for f=0, 20, ..., 300*/}
    \DataTypeTok{int}\NormalTok{ lower, upper, step;}
    \DataTypeTok{float}\NormalTok{ fahr, celsius;}

\NormalTok{    lower = }\DecValTok{0}\NormalTok{;}
\NormalTok{    upper = }\DecValTok{300}\NormalTok{;}
\NormalTok{    step = }\DecValTok{20}\NormalTok{;}

\NormalTok{    fahr = lower;}
    \ControlFlowTok{while}\NormalTok{ (fahr \textless{}= upper) \{}
\NormalTok{        celsius = (}\FloatTok{5.0}\NormalTok{/}\FloatTok{9.0}\NormalTok{) * (fahr{-}}\FloatTok{32.0}\NormalTok{);}
\NormalTok{        printf(}\StringTok{"\%4.0f \%6.1f}\SpecialCharTok{\textbackslash{}n}\StringTok{"}\NormalTok{, fahr, celsius);}
\NormalTok{        fahr = fahr + step;}
\NormalTok{    \}}
\NormalTok{\}}
\end{Highlighting}
\end{Shaded}

\begin{verbatim}
## gcc -std=gnu99 -I"/usr/share/R/include" -DNDEBUG      -fpic  -g -O2 -fdebug-prefix-map=/build/r-base-8T8CYO/r-base-4.0.3=. -fstack-protector-strong -Wformat -Werror=format-security -Wdate-time -D_FORTIFY_SOURCE=2 -g  -c c6f3f81d845528.c -o c6f3f81d845528.o
## gcc -std=gnu99 -shared -L/usr/lib/R/lib -Wl,-Bsymbolic-functions -Wl,-z,relro -o c6f3f81d845528.so c6f3f81d845528.o -L/usr/lib/R/lib -lR
\end{verbatim}

\hypertarget{receiving-input}{%
\section{Receiving Input}\label{receiving-input}}

\begin{itemize}
\tightlist
\item
  \texttt{printf}: Print to console
\item
  \texttt{getchar}: Get a string input from the console
\item
  \texttt{putchar}: Print a single character to the screen.
\end{itemize}

The program below will stop and print 'Enter something\ldots` and then received a string from the user. The while loop iterates over each letter in the input.

\begin{Shaded}
\begin{Highlighting}[]
\PreprocessorTok{\#include }\ImportTok{\textless{}stdio.h\textgreater{}}

\DataTypeTok{int}\NormalTok{ main()}
\NormalTok{\{}
    \DataTypeTok{int}\NormalTok{ c;}

\NormalTok{    printf(}\StringTok{"Enter something, or ctrl{-}D to quit.}\SpecialCharTok{\textbackslash{}n}\StringTok{"}\NormalTok{);}
\NormalTok{    c = getchar();}
    \ControlFlowTok{while}\NormalTok{ (c != EOF) \{}
\NormalTok{        printf(}\StringTok{"}\SpecialCharTok{\textbackslash{}n}\StringTok{"}\NormalTok{);}
\NormalTok{        putchar(c);}
\NormalTok{        c = getchar();}
\NormalTok{    \}}
\NormalTok{\}}
\end{Highlighting}
\end{Shaded}

\begin{verbatim}
## gcc -std=gnu99 -I"/usr/share/R/include" -DNDEBUG      -fpic  -g -O2 -fdebug-prefix-map=/build/r-base-8T8CYO/r-base-4.0.3=. -fstack-protector-strong -Wformat -Werror=format-security -Wdate-time -D_FORTIFY_SOURCE=2 -g  -c c6f3f8339b02f9.c -o c6f3f8339b02f9.o
## gcc -std=gnu99 -shared -L/usr/lib/R/lib -Wl,-Bsymbolic-functions -Wl,-z,relro -o c6f3f8339b02f9.so c6f3f8339b02f9.o -L/usr/lib/R/lib -lR
\end{verbatim}

Simpler version of the above, showing that we can use assignment within a loop.

\begin{Shaded}
\begin{Highlighting}[]
\PreprocessorTok{\#include }\ImportTok{\textless{}stdio.h\textgreater{}}

\DataTypeTok{int}\NormalTok{ main()}
\NormalTok{\{}
    \DataTypeTok{int}\NormalTok{ c;}
    
    \ControlFlowTok{while}\NormalTok{ ((c = getchar()) != EOF) }
\NormalTok{        putchar(c);}
\NormalTok{\}}
\end{Highlighting}
\end{Shaded}

\begin{verbatim}
## gcc -std=gnu99 -I"/usr/share/R/include" -DNDEBUG      -fpic  -g -O2 -fdebug-prefix-map=/build/r-base-8T8CYO/r-base-4.0.3=. -fstack-protector-strong -Wformat -Werror=format-security -Wdate-time -D_FORTIFY_SOURCE=2 -g  -c c6f3f85fba0b4a.c -o c6f3f85fba0b4a.o
## gcc -std=gnu99 -shared -L/usr/lib/R/lib -Wl,-Bsymbolic-functions -Wl,-z,relro -o c6f3f85fba0b4a.so c6f3f85fba0b4a.o -L/usr/lib/R/lib -lR
\end{verbatim}

\begin{verbatim}
## For Loop

```c
#include <stdio.h>

int main()
{
    int fahr;

    for (fahr = 0; fahr <= 300; fahr = fahr + 20)
        printf("%3d %6.1f\n", fahr, (5.0/9.0)*(fahr-32));
}
\end{verbatim}

\begin{verbatim}
## gcc -std=gnu99 -I"/usr/share/R/include" -DNDEBUG      -fpic  -g -O2 -fdebug-prefix-map=/build/r-base-8T8CYO/r-base-4.0.3=. -fstack-protector-strong -Wformat -Werror=format-security -Wdate-time -D_FORTIFY_SOURCE=2 -g  -c c6f3f84844925a.c -o c6f3f84844925a.o
## gcc -std=gnu99 -shared -L/usr/lib/R/lib -Wl,-Bsymbolic-functions -Wl,-z,relro -o c6f3f84844925a.so c6f3f84844925a.o -L/usr/lib/R/lib -lR
\end{verbatim}

\hypertarget{define}{%
\section{DEFINE}\label{define}}

\begin{Shaded}
\begin{Highlighting}[]
\PreprocessorTok{\#include }\ImportTok{\textless{}stdio.h\textgreater{}}

\PreprocessorTok{\#define LOWER 0}
\PreprocessorTok{\#define UPPER 300}
\PreprocessorTok{\#define STEP 20}

\DataTypeTok{int}\NormalTok{ main()}
\NormalTok{\{}
    \DataTypeTok{int}\NormalTok{ fahr;}

    \ControlFlowTok{for}\NormalTok{ (fahr = LOWER; fahr \textless{}= UPPER; fahr = fahr + STEP)}
\NormalTok{        printf(}\StringTok{"\%3d \%6.1f}\SpecialCharTok{\textbackslash{}n}\StringTok{"}\NormalTok{, fahr, (}\FloatTok{5.0}\NormalTok{/}\FloatTok{9.0}\NormalTok{)*(fahr{-}}\DecValTok{32}\NormalTok{));}
\NormalTok{\}}
\end{Highlighting}
\end{Shaded}

\begin{verbatim}
## gcc -std=gnu99 -I"/usr/share/R/include" -DNDEBUG      -fpic  -g -O2 -fdebug-prefix-map=/build/r-base-8T8CYO/r-base-4.0.3=. -fstack-protector-strong -Wformat -Werror=format-security -Wdate-time -D_FORTIFY_SOURCE=2 -g  -c c6f3f8287ee5cd.c -o c6f3f8287ee5cd.o
## gcc -std=gnu99 -shared -L/usr/lib/R/lib -Wl,-Bsymbolic-functions -Wl,-z,relro -o c6f3f8287ee5cd.so c6f3f8287ee5cd.o -L/usr/lib/R/lib -lR
\end{verbatim}

\hypertarget{precedence-of-operators}{%
\section{Precedence of Operators}\label{precedence-of-operators}}

\texttt{c\ =\ getchar()\ !=\ EOF} is equal to \texttt{c\ =\ (getchar()\ !=\ EOF)}

\hypertarget{charcter-count}{%
\section{Charcter Count}\label{charcter-count}}

\begin{Shaded}
\begin{Highlighting}[]
\PreprocessorTok{\#include }\ImportTok{\textless{}stdio.h\textgreater{}}

\DataTypeTok{int}\NormalTok{ main()}
\NormalTok{\{}
    \DataTypeTok{long}\NormalTok{ nc = }\DecValTok{0}\NormalTok{;}
    \ControlFlowTok{while}\NormalTok{ (getchar() != EOF)}
\NormalTok{    \{}
\NormalTok{        nc = nc + }\DecValTok{1}\NormalTok{;}
\NormalTok{        printf(}\StringTok{"There are \%ld characters}\SpecialCharTok{\textbackslash{}n}\StringTok{"}\NormalTok{, nc);}
\NormalTok{    \}}
\NormalTok{\}}
\end{Highlighting}
\end{Shaded}

\begin{verbatim}
## gcc -std=gnu99 -I"/usr/share/R/include" -DNDEBUG      -fpic  -g -O2 -fdebug-prefix-map=/build/r-base-8T8CYO/r-base-4.0.3=. -fstack-protector-strong -Wformat -Werror=format-security -Wdate-time -D_FORTIFY_SOURCE=2 -g  -c c6f3f8752e159f.c -o c6f3f8752e159f.o
## gcc -std=gnu99 -shared -L/usr/lib/R/lib -Wl,-Bsymbolic-functions -Wl,-z,relro -o c6f3f8752e159f.so c6f3f8752e159f.o -L/usr/lib/R/lib -lR
\end{verbatim}

\hypertarget{line-counting}{%
\section{Line Counting}\label{line-counting}}

\begin{Shaded}
\begin{Highlighting}[]
\PreprocessorTok{\#include }\ImportTok{\textless{}stdio.h\textgreater{}}

\DataTypeTok{int}\NormalTok{ main()}
\NormalTok{\{}
    \DataTypeTok{int}\NormalTok{ c, n1;}

\NormalTok{    n1 = }\DecValTok{0}\NormalTok{;}
    \ControlFlowTok{while}\NormalTok{((c = getchar()) != EOF)}
\NormalTok{    \{}
        \ControlFlowTok{if}\NormalTok{ (c == }\CharTok{\textquotesingle{}\textbackslash{}n\textquotesingle{}}\NormalTok{)}
\NormalTok{        \{}
\NormalTok{            printf(}\StringTok{"\%d}\SpecialCharTok{\textbackslash{}n}\StringTok{"}\NormalTok{, n1);}
\NormalTok{            ++n1;}
\NormalTok{        \}}
\NormalTok{    \}}
\NormalTok{\}}
\end{Highlighting}
\end{Shaded}

\begin{verbatim}
## gcc -std=gnu99 -I"/usr/share/R/include" -DNDEBUG      -fpic  -g -O2 -fdebug-prefix-map=/build/r-base-8T8CYO/r-base-4.0.3=. -fstack-protector-strong -Wformat -Werror=format-security -Wdate-time -D_FORTIFY_SOURCE=2 -g  -c c6f3f81915ce76.c -o c6f3f81915ce76.o
## gcc -std=gnu99 -shared -L/usr/lib/R/lib -Wl,-Bsymbolic-functions -Wl,-z,relro -o c6f3f81915ce76.so c6f3f81915ce76.o -L/usr/lib/R/lib -lR
\end{verbatim}

\hypertarget{modern-c-example}{%
\section{Modern C Example}\label{modern-c-example}}

\begin{Shaded}
\begin{Highlighting}[]
\PreprocessorTok{\#include }\ImportTok{\textless{}stdlib.h\textgreater{}}
\PreprocessorTok{\#include }\ImportTok{\textless{}stdio.h\textgreater{}}

\DataTypeTok{int}\NormalTok{ main(}\DataTypeTok{void}\NormalTok{) \{}
    \DataTypeTok{double}\NormalTok{ A[}\DecValTok{5}\NormalTok{] = \{}
\NormalTok{            [}\DecValTok{0}\NormalTok{] = }\FloatTok{9.0}\NormalTok{,}
\NormalTok{            [}\DecValTok{1}\NormalTok{] = }\FloatTok{2.9}\NormalTok{,}
\NormalTok{            [}\DecValTok{4}\NormalTok{] = }\FloatTok{3.E+25}\NormalTok{,}
\NormalTok{            [}\DecValTok{3}\NormalTok{] = }\FloatTok{0.00007}
\NormalTok{    \};}

    \ControlFlowTok{for}\NormalTok{ (}\DataTypeTok{size\_t}\NormalTok{ i = }\DecValTok{0}\NormalTok{; i \textless{} }\DecValTok{5}\NormalTok{; ++i) \{}
\NormalTok{        printf(}\StringTok{"element \%zu is \%g, }\SpecialCharTok{\textbackslash{}t}\StringTok{its square is \%g}\SpecialCharTok{\textbackslash{}n}\StringTok{"}\NormalTok{,}
\NormalTok{               i,}
\NormalTok{               A[i],}
\NormalTok{               A[i]*A[i]);}
\NormalTok{    \}}

    \ControlFlowTok{return}\NormalTok{ EXIT\_SUCCESS;}
\NormalTok{\}}
\end{Highlighting}
\end{Shaded}

\begin{verbatim}
## gcc -std=gnu99 -I"/usr/share/R/include" -DNDEBUG      -fpic  -g -O2 -fdebug-prefix-map=/build/r-base-8T8CYO/r-base-4.0.3=. -fstack-protector-strong -Wformat -Werror=format-security -Wdate-time -D_FORTIFY_SOURCE=2 -g  -c c6f3f8600c78b6.c -o c6f3f8600c78b6.o
## gcc -std=gnu99 -shared -L/usr/lib/R/lib -Wl,-Bsymbolic-functions -Wl,-z,relro -o c6f3f8600c78b6.so c6f3f8600c78b6.o -L/usr/lib/R/lib -lR
\end{verbatim}

\hypertarget{parsing-a-csv-file}{%
\section{Parsing a CSV File}\label{parsing-a-csv-file}}

\begin{Shaded}
\begin{Highlighting}[]
\PreprocessorTok{\#include }\ImportTok{\textless{}stdio.h\textgreater{}}

\PreprocessorTok{\#define ANSI\_COLOR\_PRIMARY   "\textbackslash{}x1b[32m"}

\DataTypeTok{void}\NormalTok{ print\_color(}\DataTypeTok{char}\NormalTok{* message, }\DataTypeTok{char}\NormalTok{* color) \{}
\NormalTok{    printf(}\StringTok{"\%s"}\NormalTok{, color);}
\NormalTok{    printf(}\StringTok{"\%s"}\NormalTok{, message);}
\NormalTok{    printf(}\StringTok{"\%s"}\NormalTok{, }\StringTok{"}\SpecialCharTok{\textbackslash{}x1b}\StringTok{[0m"}\NormalTok{);}
\NormalTok{\}}

\DataTypeTok{int}\NormalTok{ main(}\DataTypeTok{void}\NormalTok{) \{}
    \DataTypeTok{int}\NormalTok{ c;}

    \DataTypeTok{FILE}\NormalTok{ *file;}

\NormalTok{    file = fopen(}\StringTok{"/home/fdrennan/Programming/C/modernc/mtcars.csv"}\NormalTok{, }\StringTok{"r"}\NormalTok{);}

    \DataTypeTok{int}\NormalTok{ n\_lines = }\DecValTok{0}\NormalTok{;}
    \DataTypeTok{int}\NormalTok{ n\_columns = }\DecValTok{0}\NormalTok{;}

    \ControlFlowTok{if}\NormalTok{ (file) \{}
        \ControlFlowTok{while}\NormalTok{ ((c = getc(file)) != EOF)}
\NormalTok{        \{}
            \ControlFlowTok{if}\NormalTok{ (c == }\CharTok{\textquotesingle{}\textbackslash{}n\textquotesingle{}}\NormalTok{)}
\NormalTok{                ++n\_lines;}
            \ControlFlowTok{if}\NormalTok{ ((n\_lines \textless{} }\DecValTok{1}\NormalTok{) \& (c == }\CharTok{\textquotesingle{},\textquotesingle{}}\NormalTok{))}
\NormalTok{                ++n\_columns;}
            \ControlFlowTok{if}\NormalTok{ (n\_lines \textless{} }\DecValTok{5}\NormalTok{) \{}
\NormalTok{                putc(c, stdout);}
                \ControlFlowTok{if}\NormalTok{(c == }\CharTok{\textquotesingle{},\textquotesingle{}}\NormalTok{)}
\NormalTok{                    printf(}\StringTok{" "}\NormalTok{);}
\NormalTok{            \}}
\NormalTok{        \}}
\NormalTok{        fclose(file);}
\NormalTok{    \}}

\NormalTok{    print\_color(}\StringTok{"}\SpecialCharTok{\textbackslash{}n\textbackslash{}n}\StringTok{Number of lines: "}\NormalTok{, ANSI\_COLOR\_PRIMARY);}
\NormalTok{    printf(}\StringTok{"\%d}\SpecialCharTok{\textbackslash{}n}\StringTok{"}\NormalTok{, n\_lines {-} }\DecValTok{1}\NormalTok{);}

\NormalTok{    print\_color(}\StringTok{"Number of columns: "}\NormalTok{, ANSI\_COLOR\_PRIMARY);}
\NormalTok{    printf(}\StringTok{"\%d}\SpecialCharTok{\textbackslash{}n}\StringTok{"}\NormalTok{, n\_columns);}

\NormalTok{\}}
\end{Highlighting}
\end{Shaded}

\begin{verbatim}
## gcc -std=gnu99 -I"/usr/share/R/include" -DNDEBUG      -fpic  -g -O2 -fdebug-prefix-map=/build/r-base-8T8CYO/r-base-4.0.3=. -fstack-protector-strong -Wformat -Werror=format-security -Wdate-time -D_FORTIFY_SOURCE=2 -g  -c c6f3f8561e1bd3.c -o c6f3f8561e1bd3.o
## gcc -std=gnu99 -shared -L/usr/lib/R/lib -Wl,-Bsymbolic-functions -Wl,-z,relro -o c6f3f8561e1bd3.so c6f3f8561e1bd3.o -L/usr/lib/R/lib -lR
\end{verbatim}

\hypertarget{c-1}{%
\chapter{C++}\label{c-1}}

Here is a review of existing methods.

\hypertarget{rust}{%
\chapter{Rust}\label{rust}}

\begin{verbatim}
fn main () 
{
  println!("Yo, this is cool.");
}
\end{verbatim}

  \bibliography{book.bib,packages.bib}

\end{document}
